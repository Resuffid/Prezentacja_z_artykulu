\documentclass[pdf]{beamer}
\usepackage{times}
\usepackage{polski}
\usepackage[utf8]{inputenc}
\usepackage[T1]{fontenc}
\usetheme{Pittsburgh}
\usecolortheme{wolverine}
\usefonttheme{serif}
\mode<presentation>{}
\title{Rola platformy Instagram w odbiorze własnego wyglądu:}
\subtitle{Badanie zależności między modyfikowaniem swojego wizerunku, a oceną postrzegania swojego ciała wśród kobiet}
\author{Sara Pasturczak, Aleksandra Popowska, Wiktor Warchałowski\\Gdański Uniwersytet Medyczny}
\date{24 Maja 2022}

\begin{document}

\begin{frame}
  \titlepage
\end{frame}

\begin{frame}{Wprowadzenie}
\begin{center}
Dotychczasowe badania przedstawiały różne, sprzeczne ze sobą wyniki, iż platformy społecznościowe zwiększają, zmniejszają lub w ogóle nie wpływają na samoakceptację. Z tego powodu, celem niniejszego artykułu jest sprawdzenie czy fakt modyfikacji zdjęć własnego wizerunku na platformie Instagram ma związek z poziomem zadowolenia z własnego ciała u młodych kobiet
\end{center}
\end{frame}

\end{document}